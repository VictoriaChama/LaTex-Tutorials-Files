\documentclass{article}
\usepackage{amsmath}
\usepackage{algorithmic}
\usepackage{algorithm}


\title{Equations and Math Symbols}
\author{Victoria Chama}

\begin{document}
	\maketitle
	
	\section{First Equations - Inline}
	
	This equation is inline $b = 5*4$.
	This equation is also inline $b = 5^4$.
	
	\section{With Floating Environment}
	

	\begin{equation}
	\label{equation_1}
	\alpha \,\, \leq  \,\, \beta \,\, \geq \,\, \rightarrow \,\, \text{ To add space at the begining and at the end of the document } \,\, \cup 
	\end{equation}
	
	
		% So if you do not want the equation number to be added for you use the \begin{equation*} environment which can be used by added the package amsamath
	\begin{equation*}
	5*4 = 20
	\end{equation*}
	
	This is the reference to the equation in the Text. See Equation ~\ref{equation_1}
	
	\section{Adding Multiple Arrays}
	Sometimes you want to add multiple arrays using one environment because they are similar. to do that use the begin{eqnarray} environment
	 
	\begin{eqnarray}
	\label{equation_2}
	5 + 5= 10 \\ 
	\label{equation_3}
	7 + 2 + 70 &= 867
	\end{eqnarray}
	
	This is Equation ~\ref{equation_2} and this is equation ~\ref{equation_3}. To reinforce alignment you can also use the AND sign and LaTex will try its best to realign based on the position of the AND symbol. You can also use 2 AND signs to add more reinforcement. 
	
	\section{How to include Matrices0 in your Document}
	
	This section illustrates how you include matrices to the document.
	Curved brackets
	% Square brackets environment are used for the matrices
	\[
	 M = \begin{pmatrix}
	 1 & 3 & 78 \\
	 29 & 8 & 10 \\
	 78 & 2 & 2 
	 \end{pmatrix}
	\]
	
	Square brackets
	\[
	M = \begin{bmatrix}
	1 & 3 & 78 \\
	29 & 8 & 10 \\
	78 & 2 & 2 
	\end{bmatrix}
	\]
	
	\section{Working With Algorithms}
	In this section of the tutorials we look at hoew to format algorithms using LaTex. See Algorithm~\ref{algorithm_1} below:
	
	\begin{algorithm}[htbp]
		\caption{This is our Sample Algorithm}
		\label{algorithm_1}
	\begin{algorithmic}
		\renewcommand{\algorithmicrequire}{\textbf{Input:}}
		\renewcommand{\algorithmicensure}{\textbf{Output:}}
		\ENSURE The Corresponding
		\REQUIRE Information Table
		\IF {$ a \geq 5$}
			\STATE The Value is greater than Five.
			
		\ELSE
			\WHILE {a != 5}
				\STATE A is not equal to Five
			\ENDWHILE
		\ENDIF 	
	\end{algorithmic}
	\end{algorithm}
	
	
\end{document}